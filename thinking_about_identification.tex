\documentclass[a4paper,12pt,twoside]{article}

\usepackage[utf8]{inputenc}
\usepackage[T2A]{fontenc}
\usepackage[english]{babel}
\usepackage{booktabs}
\usepackage[margin = 1in,includeheadfoot]{geometry}
\usepackage{graphicx}
\usepackage{amsmath}
\usepackage{amssymb}
\usepackage{tikz}
\usepackage{fancyhdr}

\parindent=0pt
\parskip=10pt

\pagestyle{fancy}
\fancyhf{}
\rhead{Alexander Petrov}
\lhead{}
\rfoot{\thepage}

%hyperlinks package -- should be the last to import
\usepackage{hyperref}
\hypersetup{
	colorlinks = true,
	linkcolor=blue,
	citecolor=blue,
	urlcolor=blue }

\begin{document}

Parameter of interest:

\begin{equation}
	(Y(1), Y(0), D) \sim F
\end{equation}

Observed vector:

\begin{equation}
	(D Y(1) + (1 - D) Y(0), D) \sim G
\end{equation}

Consider the case when $P(D = 1) = 0$. Does it necessarily imply that $F$ cannot be identified? I think so now\dots

But my question was inspired by the following consideration: Take the subset of the sample space $S = \{ \omega \in \Omega : D(\omega) = 0\}$. We started with the assumption that $P(S) = 0$. I was thinking that maybe one can still map $S$ into all possible values of $Y(0)$ in a way that would still make identification possible. However, the thing is that any subset of a measure $0$ set has to have measure $0$ also, damn... Therefore, whatever way we map elements of $S$ into the range of $Y(0)$, $P(Y(0) < y | D = 0) = 0$, so the conditional distribution... wait WHAT THE FUCK, how do conditional distributions work if the probability of any variable taking a certain value is $0$??? geezus... brb

\end{document}